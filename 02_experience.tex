\section*{\workExperienceTitle} % Título de la sección de experiencia laboral, definido en el archivo de traducciones
\vspace{-20pt}
\noindent\makebox[\linewidth]{\rule{\textwidth}{0.4pt}} % Línea horizontal debajo del título
\vspace{-10pt}

% Configuración de la alineación del texto
\setlength{\leftskip}{0pt} % Sin sangría a la izquierda
\setlength{\rightskip}{0pt plus 1fil} % Sangría flexible a la derecha para justificación
\raggedright % Alineación del texto a la izquierda

\begin{luacode}
-- Cargar el archivo JSON
local file = io.open('02_experience.json', 'r') -- Abre el archivo JSON que contiene los datos de experiencia laboral
local jsonstring = file:read('*a') -- Lee todo el contenido del archivo como una cadena
file:close() -- Cierra el archivo

-- Verificar si se cargó correctamente
if not jsonstring then
    tex.print("\\textbf{Error: No se pudo cargar el archivo JSON.}") -- Mensaje de error si el archivo no se pudo cargar
    return
end

-- Convertir JSON a tabla Lua
local jsondata = utilities.json.tolua(jsonstring) -- Convierte la cadena JSON a una tabla de Lua

-- Verificar si el JSON se convirtió correctamente
if not jsondata then
    tex.print("\\textbf{Error: El archivo JSON no es válido.}") -- Mensaje de error si el archivo JSON no es válido
    return
end

-- Definir el idioma y la versión del CV
local lang = "\cvLang" -- Variable para el idioma seleccionado
local version = "\cvVersion" -- Variable para la versión del CV (extenso o resumido)

-- Función para formatear las fechas de la experiencia
local function formatDate(start_month, start_year, end_month, end_year)
    local date_line = ""

    if start_month ~= "" and end_month ~= "" then
        if start_year == end_year then
            if start_month == end_month then
                date_line = "\\getMonthName{" .. start_month .. "}\\unskip, " .. start_year
            else
                date_line = "\\getMonthName{" .. start_month .. "} -- \\getMonthName{" .. end_month .. "}\\unskip, " .. start_year
            end
        else
            date_line = "\\getMonthName{" .. start_month .. "}\\unskip, " .. start_year .. " -- \\getMonthName{" .. end_month .. "}\\unskip, " .. end_year
        end
    elseif start_year ~= "" and end_year ~= "" then
        if start_year == end_year then
            date_line = start_year
        else
            date_line = start_year .. " -- " .. end_year
        end
    elseif start_year ~= "" then
        date_line = start_year .. " -- \\presente" -- Muestra "Presente" si no hay fecha de término
    end

    return date_line
end

-- Ordenar por año de término, y por fecha de inicio en caso de empate
table.sort(jsondata.experience, function(a, b)
    local a_end_year = (a.date and a.date.end_year ~= "" and a.date.end_year) or 9999 -- Asigna 9999 si no hay fecha de término
    local b_end_year = (b.date and b.date.end_year ~= "" and b.date.end_year) or 9999

    if tonumber(a_end_year) == tonumber(b_end_year) then
        local a_start_year = (a.date and a.date.start_year ~= "" and a.date.start_year) or 0 -- Asigna 0 si no hay fecha de inicio
        local b_start_year = (b.date and b.date.start_year ~= "" and b.date.start_year) or 0
        return tonumber(a_start_year) > tonumber(b_start_year) -- Ordena por fecha de inicio si las fechas de término son iguales
    else
        return tonumber(a_end_year) > tonumber(b_end_year)
    end
end)

-- Variables para controlar la agrupación
local current_institution = ""

-- Iterar sobre los datos de experiencia y mostrarlos
for _, exp in ipairs(jsondata.experience) do
    -- Verificar si la experiencia debe mostrarse en la versión actual del CV
    if exp.show_in == "both" or exp.show_in == version then
        local title = exp.title[lang] or "Title not specified" -- Obtiene el título en el idioma seleccionado
        local institution = exp.institution[lang] or "Institution not specified" -- Obtiene la institución en el idioma seleccionado
        local location = exp.location and exp.location[lang] or "" -- Obtiene la ubicación en el idioma seleccionado
        local start_year = exp.date and exp.date.start_year or "" -- Obtiene el año de inicio
        local end_year = exp.date and exp.date.end_year or "" -- Obtiene el año de término
        local start_month = exp.date and exp.date.start_month or "" -- Obtiene el mes de inicio
        local end_month = exp.date and exp.date.end_month or "" -- Obtiene el mes de término

        -- Usar la nueva función para formatear la línea de fecha
        local date_line = formatDate(start_month, start_year, end_month, end_year)

        -- Verificar si la institución ha cambiado
        if institution ~= current_institution then
            if current_institution ~= "" then
                tex.print("\\par\\medskip") -- Añadir espacio entre instituciones
            end
            tex.print("\\noindent\\textbf{" .. institution .. "} \\hfill \\textit{" .. location .. "} \\\\ \\medskip")
            current_institution = institution -- Actualiza la institución actual
        end

        -- Imprimir el título y la fecha
        tex.print("\\noindent\\textit{" .. title .. "} \\hfill \\textit{" .. date_line .. "} \\\\ \\smallskip")

        -- Iterar sobre las descripciones y mostrarlas si corresponden a la versión actual
        if exp.descriptions then
            for _, description in ipairs(exp.descriptions) do
                if description.show_in == "both" or description.show_in == version then
                    local desc_text = description[lang] or "Description not specified"
                    tex.print("{\\footnotesize \\noindent\\begin{quote}" .. desc_text .. "\\end{quote}}\\\\ \\medskip") -- Espaciado ajustado con indentación
                end
            end
        end
    end
end
\end{luacode}
