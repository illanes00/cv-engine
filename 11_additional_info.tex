\section*{\additionalInfoTitle} % Título de la sección de información adicional, definido en el archivo de traducciones
\vspace{-20pt}
\noindent\makebox[\linewidth]{\rule{\textwidth}{0.4pt}} % Línea horizontal debajo del título
\vspace{-10pt}

% Configuración de alineación del texto
\setlength{\leftskip}{0pt} % Sin sangría a la izquierda
\setlength{\rightskip}{0pt plus 1fil} % Sangría flexible a la derecha para justificación
\raggedright % Alineación del texto a la izquierda

\begin{luacode}
-- Cargar el archivo JSON
local file = io.open('11_additional_info.json', 'r') -- Abre el archivo JSON que contiene la información adicional
local jsonstring = file:read('*a') -- Lee todo el contenido del archivo como una cadena
file:close() -- Cierra el archivo

-- Verificar si la cadena JSON es válida
if not jsonstring or jsonstring == "" then
    tex.print("\\textbf{Error: No se pudo cargar el archivo JSON.}") -- Mensaje de error si no se puede cargar el archivo
    return
end

-- Convertir la cadena JSON a una tabla Lua
local jsondata = utilities.json.tolua(jsonstring) -- Convierte la cadena JSON a una tabla de Lua

-- Definir el idioma y la versión del CV
local lang = "\cvLang" -- Variable para el idioma seleccionado
local version = "\cvVersion" -- Variable para la versión del CV (extenso o resumido)

-- Iterar sobre los datos de información adicional
for _, info in ipairs(jsondata.additional_info) do
    if info.show_in == "both" or info.show_in == version then
        local title = info.title[lang] or "Título no especificado" -- Obtiene el título en el idioma seleccionado
        local content = info.content[lang] or "Contenido no especificado" -- Obtiene el contenido en el idioma seleccionado

        -- Imprimir el título y el contenido con el formato adecuado
        tex.print("\\noindent\\textbf{" .. title:gsub("_", "\\_") .. "}: " .. content:gsub("_", "\\_") .. " \\\\ \\vspace{8pt}")
    end
end
\end{luacode}
